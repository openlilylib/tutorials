%%%%%%%%%%%%%%%%%%%%%%%%%%%%%%%%%%%%%%%%%%%%%%%%%%%%%%%%%%%%%%%%%%%%%%%%%%%
%                                                                         %
%      This file is part of the 'openLilyLib' library.                    %
%                                ===========                              %
%                                                                         %
%              https://github.com/lilyglyphs/openLilyLib                  %
%                                                                         %
%  Copyright 2012-13 by Urs Liska, lilyglyphs@ursliska.de                 %
%                                                                         %
%  'openLilyLib' is free software: you can redistribute it and/or modify  %
%  it under the terms of the GNU General Public License as published by   %
%  the Free Software Foundation, either version 3 of the License, or      %
%  (at your option) any later version.                                    %
%                                                                         %
%  This program is distributed in the hope that it will be useful,        %
%  but WITHOUT ANY WARRANTY; without even the implied warranty of         %
%  MERCHANTABILITY or FITNESS FOR A PARTICULAR PURPOSE. See the           %
%  GNU General Public License for more details.                           %
%                                                                         %
%  You should have received a copy of the GNU General Public License      %
%  along with this program.  If not, see <http://www.gnu.org/licenses/>.  %
%                                                                         %
%%%%%%%%%%%%%%%%%%%%%%%%%%%%%%%%%%%%%%%%%%%%%%%%%%%%%%%%%%%%%%%%%%%%%%%%%%%

\documentclass[../../LilyPond-Tutorials]{subfiles}

\begin{document}
\parttitle[Urs Liska]{Editing Musical Documents as Plain Text}
\begin{authorAbstract}{Urs Liska}
Abstract
\end{authorAbstract}

\chapter*{Introduction}
\label{chap:introduction}
This paper discusses an approach to authoring musical documents%
\footnote{i.\,e.\ scores and texts about music}
that is based on \emph{plain text files} as opposed to those using graphical \textsc{wysiwyg} software.
These concepts, tools, and workflows have significantly changed my life as a document author, and I  wholeheartedly endorse them because I strongly believe in their unique and substantial advantages.

The plain text approach is practically non-existent in the humanist disciplines or in the music business, while being de facto standard in many natural and computer sciences.
Working with plain text based tools indeed requires a certain shift in mind-set for people who aren't already familiar with the working paradigms through their profession.
And it can't be denied that the learning curve is considerable.
But this investment is absolutely justified because on the long run it greatly benefits productivity and offers potentials unimaginable otherwise.
Reading and writing music, playing an instrument, investigating a manuscript source---all this involves a very long and intense learning curve, and we mastered them as a matter of course in order to become the professionals we are.
So why be afraid learning something new?

This paper is focused on scholarly and collaborative workflows, particularly preparing musical editions for publication, because that's my cup or tea.
But most of it will equally apply to creating musical scores or texts about music that have a certain level of complexity.
The described concepts will also prove useful for people who have repeatedly to do with comparable documents, such as presentations, teaching/exam materials, music examples etc.
However I will \emph{not} cover workflows that mainly depend on instant accessibility or results but don't care about structure or output quality, such as just-in-time arranging or the like.
For such applications plain text tools may not be the appropriate choice.

If you expect an in-depth description or even a guide how to use the tools I describe you might be disappointed because that's not the intention of this document.
What I would like to achieve is giving you a sense of the power that text based approaches can give you.
For this I will discuss most of it from a rather elevated point of view, keeping your exposure to concrete examples at a minimum.
At the end of the text I will direct you to more extensive material that is suitable to get you more intimately acquainted with the concepts and that may actually get you in touch with editing plain text files.

The programs I will introduce to you on the following pages are:
\begin{itemize*}
\item \emph{LilyPond} -- the program that lets you engrave beautiful scores
\item \emph{Git} -- the versioning system that keeps your work under control
\item \emph{\LaTeX} -- the professional typesetting engine for text documents
\end{itemize*}


\end{document}