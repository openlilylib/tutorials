%%%%%%%%%%%%%%%%%%%%%%%%%%%%%%%%%%%%%%%%%%%%%%%%%%%%%%%%%%%%%%%%%%%%%%%%%%%
%                                                                         %
%      This file is part of the 'openLilyLib' library.                    %
%                                ===========                              %
%                                                                         %
%              https://github.com/lilyglyphs/openLilyLib                  %
%                                                                         %
%  Copyright 2012-13 by Urs Liska, lilyglyphs@ursliska.de                 %
%                                                                         %
%  'openLilyLib' is free software: you can redistribute it and/or modify  %
%  it under the terms of the GNU General Public License as published by   %
%  the Free Software Foundation, either version 3 of the License, or      %
%  (at your option) any later version.                                    %
%                                                                         %
%  This program is distributed in the hope that it will be useful,        %
%  but WITHOUT ANY WARRANTY; without even the implied warranty of         %
%  MERCHANTABILITY or FITNESS FOR A PARTICULAR PURPOSE. See the           %
%  GNU General Public License for more details.                           %
%                                                                         %
%  You should have received a copy of the GNU General Public License      %
%  along with this program.  If not, see <http://www.gnu.org/licenses/>.  %
%                                                                         %
%%%%%%%%%%%%%%%%%%%%%%%%%%%%%%%%%%%%%%%%%%%%%%%%%%%%%%%%%%%%%%%%%%%%%%%%%%%

% 
%
%%%%%%%%%%%%%%%%%%%%%%%%%%%%%%%%%%%%%%%%%%%%%%%%%%%%%%%%%%%%%%%%%%%%%%%%%%%

\documentclass[../../LilyPond-Tutorials]{subfiles}

\begin{document}

\parttitle{Writing (About) Music with Plain-Text Based Tools}
\begin{authorAbstract}{Urs Liska}
Abstract
\end{authorAbstract}

\chapter{Outline}

\begin{itemize}
\item Main purpose of the document
\item Intended audience?\\
-- People who write text documents about and including music\\
-- People who typeset musical scores.
\item I want to present a set of tools who work on the compilation of source files instead of binary file formats and WYSIWYG editing.
\item What do I mean with “plain-text based tools”?\\
Difference binary/plain-text file formats and write/compile vs. WYSIWYG (use of editors)
	\begin{itemize*}
	\item Human readable
	\item Maintainable (attention: Character encoding!)
	\item Machine readable, scriptable\\
	Programs can analyze or process files\\
	Programs can create files
	\item Comments can be inserted in-place
	\item Stability: Any formatting/markup/tweaks are inserted in a robust way
	\item \textbf{diffable}
	\end{itemize*}
\item diffable means ready for version control -> techniques used in software development\\
One of the major impacts is on collaboration.
\item totally different work-flow paradigm\\
It can't be denied that this requires quite a shift in the ... mind-set\\
and that the learning curve is considerable (especially if you're not a programmer type)\\
But I'd say it's an investment for your life (find good comparisons to other tools we have to learn. Maybe also a language that we'd have to learn for our job?)\\
And one starts with simple examples -- these are easy to grasp.

\item I will present four programs/systems in this paper:
	\begin{itemize*}
	\item Markdown (markup language) -- to be used for notes and sketches
	\item LaTeX (typesetting system) -- to create professional text documents
	\item LilyPond -- Music notation software
	\item Git (Distributed Version Control System) -- to keep it all together
	\end{itemize*}
\end{itemize}

\chapter{Markdown}

\begin{itemize}
\item As a gentle introduction ;-)
\item Describe HTML
	\begin{itemize*}
	\item Markup language
	\item Plain text based
	\item Separation of content and appearance
	\item Source code “readable” but not really expressive
	\end{itemize*}
\item Markdown wants to define a syntax to “markup” text in a way that the plain text nearly shows what is meant
\item Editors can display a formatted version of the text\\
Often used for Web, e.g. for entering forum entries: One enters markdown and the web server transforms that into HTML
\item Very useful for short sketches -- way less overhead than using a word processor.
\item One example document in ReText\\
Show a few syntax examples (including links)\\
Show the option to export to other formats.
\item possible editors for different OSs
\end{itemize}

\chapter{\LaTeX}
\begin{itemize}
\item Main purpose: Typesetting text documents\\
Although there are powerful capabilites in other graphical areas.
\item History, architecture
\item Document setup
\item syntax at its most basic level\\
Compare semantic and direct markup in latex and word\\
(e.g. also search\&replace)
\item Don't know how many examples (and how complex) I should provide
\item Extending the capabilites through packages\\
A few examples (e.g. mdwlist)
\item Implement your own packages/classes to hide away complexity\\
Bsp. concert programme/letter -> resulting source file is really tiny
\item Specifically musical issues
	\begin{itemize*}
	\item musicexamples\\
	Only mention its existence and show its use later (after LilyPond introduction)
	\item lilyglyphs.
	\end{itemize*}
\item Editors/IDEs\\
Concept of syntax highlighting
\end{itemize}

\chapter{LilyPond}
\begin{itemize}
\item the “\LaTeX” of music notation (even historically)
\item short history (summary/excerpts from the lilypond.org essay?) and main objective
\item Highlight the engraving quality as a result of the non-wysiwyg approach
\item Basic syntax introduction (how is music represented?)
\end{itemize}

\section*{Characteristics and Qualities of LilyPond}
\subsection*{Characteristics due to the plain-text file format}
\begin{itemize}
\item Ability to place (technical and editorial) comments in the source code
\item Source code is “robust”, i.\,e. no hidden settings\\
I can see explicitely if something has been tweaked or not\\
“Update Layout” won't revert your changes arbitraryly\\
Everything is there, written down explicitely
\item Use of variables:\\
This is transparent and robust\\
Easy transposition and “stretching”\\
\item Use of (empty) variables as switches, e.\,g. for draft/final mode
\item Variables imply huge potential for reusing code for different score versions:\\
-- Score/parts\\
-- Different layouts, e.\,g. conductor's/study score\\
-- Excerpts, e.\,g. for music examples
\item Style sheets:\\
Apply consistent settings across multiple scores (e.\,g. examples in a book)\\
Easily change settings in a central place, then recompile all examples\\
Easily switch between versions by uncommenting/commenting out include files
\item Music can be generated by programs\\
Example: Midnight-minutes\\
Example: lilyglyphs generation scripts
\item Features for theory, teaching etc.?\\
Schenker graphs\\
Empty spaces (for tests)\\
More?
\end{itemize}

\subsection*{Typical Qualities}
\begin{itemize}
\item Very good readability of the out-of-the-box results
\item Flexibility for time management
\item good control of the output format: Important for the creation of music examples
\end{itemize}

\subsection*{Working with LilyPond}
\begin{itemize}
\item Editors: There are a few, but I'll only talk about Frescobaldi
	\begin{itemize*}
	\item Collect features to be stated here
	\item TODO: fix my installation so the music view works again!!!
	\item 2-way Point-and-click
	\item development is readily available and listening
	\end{itemize*}
\item How to work with LilyPond (continue first syntax introduction)
	\begin{itemize*}
	\item Show simple tweaks (with preset commands like \cmd{dynamicsUp})
	\item Show some more complex tweaks
	\item Show a slur tweaking (with \cmd{displayControlPoints} on)
	\item More?
	\end{itemize*}
\end{itemize}

\subsection*{Integrating \LaTeX{} and LilyPond}
\begin{itemize}
\item Show the use of \texttt{musicexamples} and \texttt{lilyglyphs}
\item mention \texttt{lilypond-book}
\end{itemize}

\chapter{Git}
\begin{itemize}
\item What means “versioning”?\\
Point out the relationship with/the origin from software development
\item What is a diff?\\
Point out the plain text issue again.
\item What is a commit?\\
Important: like a fine-tuned “save”\\
Maybe mention the basic \texttt{rebase} functionality?
\item What do we gain from the commit history?
	\begin{itemize*}
	\item  A unlimited and especially a completely selectable “undo/redo” function
	\item Can inspect \emph{any} state the project has been in
	\item Can apply any single changeset\\
	(Example with Cherry-Picking)
	\end{itemize*}
\item Branching
	\begin{itemize*}
	\item Separate workspaces
	\item work without affecting the stable state
	\item easily switch between different states
	\end{itemize*}
\end{itemize}

\section{Collaboration with Git}
\subsection*{Traditional approaches}
\begin{itemize}
\item Exchanging documents by email
	\begin{itemize*}
	\item May work with \emph{very} few collaborators
	\item If we're sending documents back and forth we're getting multiple versions of the files -- at least in the mail archive, possibly also on disk $\rightarrow$ possible confusion
	\item Office packages allow to “record changes”, which allows to see what the other did, but:\\
	What if A sends B a file, B modifies and returns the file, but A has done further changes in the meantime?\\
	You'll end up comparing files manually!
	\end{itemize*}
\item Storing files in a “cloud” based location like Dropbox or another network based solution
	\begin{itemize*}
	\item Works better because there aren't multiple copies of the files\\
	(or if there are they are transparently managed by the service provider)
	\item But you still have to lock files by email, which is error prone
	\item What if I want to work offline for a while (because one of my computers doesn't have internet access or I'm in the train)?\\
	I have to “lock” all files I \emph{might} possibly work on, or just work and hope nobody else touches the files in the meantime.
	\end{itemize*}
\end{itemize}
Both approaches have severe drawbacks. And especially they become virtually un-manageable when there are more than two or three participants to be coordinated.

\subsection*{“Enter Version Control”}
\begin{itemize}
\item Local and Remote repos
\item Distributed VCS:\\
Central shared repository, (unlimited number of) independent local repositories\\
Local repo has the \emph{complete} code base\\
light-weight, small commits are encouraged\\
work offline and only now and then merge the results
\item Push and Pull (relationship commit?)\\
first pull then push
\item Due to the line-by-line comparison I don't need to lock anything\\
I just work locally and then bring the changes in line with upstream\\
Quite small risk of merge conflicts\\
If they arise they are usually simple to resolve.
\end{itemize}

\begin{itemize}
\item Service providers: GitHub and BitBucket\\
Difference mainly the pricing policy (explain)
\item Show web site (e.g. issues)
\item Some examples for commits:\\
Create one branch in the demo repository and go through them \\
Show the evolution of one file with the “view file @” button
	\begin{itemize*}
	\item Example where a commit closes an issue
	\item Single change of a note
	\item complex tweak
	\item changeset across several files
	\item “Blame” -- several contributors
	\end{itemize*}
\end{itemize}

\chapter{Conclusion}
\begin{itemize}
\item Advantages
	\begin{itemize*}
	\item History: Unlimited undo/redo\\
	One can even revert changes partially (i.e. some changes inside one commit/file)
	\item Easy collaboration on a set of files
	\end{itemize*}
\item Difficulties:
	\begin{itemize*}
	\item Learning curve
	\item One is restricted to plain text documents\\
	other file types need special consideration
	\end{itemize*}
\item Data exchange
	\begin{itemize*}
	\item Text document to LaTeX quite straightforward
	\item LaTeX document to Office document also (of course some incompatibilites (see command with arguments))
	\item scores into lilypond: Possible through MusicXML
	\item So far it isn't possible to export LilyPond scores to other formats\\
	This is a major drawback as many publishers insist on using Finale or Sibelius files (unfortunately)\\
	could be changed comparably easily.
	\end{itemize*}
\item Use cases
	\begin{itemize*}
	\item scholarly edition\\
	Different participants can work on the same set of files: Entering music --- scholarly revision --- proof-reading --- beautify engraving --- critical report --- compile final volume\\
	If for example there are musical errors to fix during the final stage (prepress) they can be just applied (in a bug-fix branch) and tested. If they produce side-effects the engraver can fix these, otherwise no further action has to be taken.
	\item “Crowd” edition\\
	There can really be many people involved in an edition (just like in a software project)\\
	This can be used to speed up the process (use the term “scalability”?), e.g.\\
	-- each participant does a few pieces/movements (e.g. in a song book) --> Wilbert?\\
	-- each participant enters a part from a score (possibly) with peer review work-flows
	\item Books\\
	Editors and proof-readers can simply work in the original document, author can (partially) accept changes.\\
	-- another use: Translations, improvemnts in the language
	\item Distinguishing between project and community members\\
	-- read-only and read/write (or push) access.\\
	-- Fork / Pull request approach (show online?)
	\end{itemize*}
\end{itemize}

\end{document}